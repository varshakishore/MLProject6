\documentclass[11pt]{article}
\usepackage[margin=1in]{geometry}
\usepackage{enumerate}
\usepackage{framed}
\usepackage{multirow}
\usepackage{multicol}
\usepackage{bm}
\usepackage{amssymb}
\usepackage{amsmath}
\usepackage{amsthm}
\usepackage{multicol}
\usepackage{graphicx}
\setlength{\columnsep}{1in}
\begin{document}

\newcommand{\Name}[1]{\noindent \textbf{Name:} #1 \\}
\newcommand{\pderiv}[2]{\frac{\partial #1}{\partial #2}}
\newcommand{\psderiv}[3]{\frac{\partial^2 #1}{\partial #2 \partial #3}}

\begin{center}
	\bf
	Machine Learning \\
	Computer Science 158 \\
	Spring 2017 \\
	\rm
	Project 6\\
	Due: March 8 at 11:59 PM \\
\end{center}
\noindent \textbf{Name: Varsha Kishore and Savannah Baron} \\
\begin{enumerate}[(2)]
\item Hyperparameter Selection
\begin{enumerate}[(b)]
\item As we discussed in class, when training using accuracy as the measure, 
the most accurate approach can be to classify everything as the majority. So, if we 
keep the proportion the same no one fold will have an advantage over the other while
training in terms of ratio of majority to minority. 
\item Code complete!
\item 
[0.6554, 0.7918, 0.8649, 0.6554, 1.0, 0.0]
[0.7749, 0.8454, 0.8654, 0.7672, 0.9427, 0.4559]
[0.8248, 0.8715, 0.8981, 0.8393, 0.9073, 0.668]
[0.8445, 0.8832, 0.9101, 0.8692, 0.8991, 0.7408]
[0.8391, 0.8796, 0.9086, 0.8623, 0.8991, 0.7252]
[0.8391, 0.8796, 0.9086, 0.8623, 0.8991, 0.7252]
MAX: [ 0.8445  0.8832  0.9101  0.8692  1.      0.7408]
\end{enumerate}
\end{enumerate}


\end{document}